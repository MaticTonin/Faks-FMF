\documentclass[11pt, a4paper]{article}
\usepackage[utf8]{inputenc}

\usepackage[margin=1in]{geometry} 
\usepackage{amsmath,amsthm,amssymb}
\usepackage[margin=1in]{geometry} 
\usepackage{amsmath,amsthm,amssymb}

\usepackage[slovene]{babel}
\usepackage{color}
\usepackage{graphicx}
\usepackage{amssymb}
\usepackage{amsmath}
\usepackage{mathtools}
\usepackage{commath}
\usepackage{ragged2e}
\usepackage[T1]{fontenc}
\usepackage[normalem]{ulem}
\usepackage{amsthm}
\usepackage{esvect}
\usepackage{float}
\usepackage{calrsfs}
\DeclareMathAlphabet{\pazocal}{OMS}{zplm}{m}{n}
\newcommand{\Ga}{\mathcal{G}}
\mathtoolsset{showonlyrefs} 

\newcommand\setItemnumber[1]{\setcounter{enumi}{\numexpr#1-1\relax}}


\newtheorem{theorem}{Trditev}[section]
\newtheorem{corollary}{Posledica}[section]
\newtheorem{lemma}[section]{Lema}
\theoremstyle{definition}
\newtheorem{definition}{Definicija}[section]
\theoremstyle{example}
\newtheorem{example}[section]{Primer}
\theoremstyle{izrek}
\newtheorem{izrek}[section]{Izrek}

\begin{document}
\begin{center}
\thispagestyle{empty}
\parskip=14pt%
\vspace*{3\parskip}%
\begin{Huge} Vrtavka \end{Huge}

By

Matic Tonin

ID No. (28181098)

Mentor 

(Rok Dolenec)

\rule{7cm}{0.4pt}

Pod okvirom:

FAKULTETE ZA FIZIKO IN MATEMATIKO, LJUBLJANA

1. 4. 2020

\end{center}
\pagebreak
\section{Naloga}
Izmeri precesijsko ($\omega_{pr}$) in nutacijsko kotno hitrost ($\omega_N$) v odvisnosti od kotne hitrosti ($\omega_z$) vrtavke. Izvedi meritev pri vsaj treh frekvencah $\nu_z = \omega_z/(2\pi)$. Na primer pri približno
600, 500 in 400 obratov na minuto (kratica rpm – angl. rotations per minute).
Gornjo meritev izvedi pri naslednjih nastavitvah vrtavke:

\begin{enumerate}
	\item vrtavka z utežjo blizu krogle
	\item utež na sredini palice
	\item utež na koncu palice (pusti si prostor za oprijem)
\end{enumerate}

Meritve z različnimi nastavitvami vrtavke izvedi pri podobnih frekvencah $\nu_z$ kot prej,
da so rezultati lažje primerljivi. Izmerjene vrednosti $\omega_{pr}$ in $\omega_N$ primerjaj z izračunanimi
iz (\ref{precesija}) in (\ref{nutacija}) in naredi tabelo.


Kljub temu, da meritve sam nisem izvajal, bom v poročilu navedel, kakšen bi moral biti postopek dela in kako smo dobili določene mertve. \\
\bigskip

\section{Postopek dela} 

Najprej smo si morali izbrati, pri 3 različne frekvence, ki jih bomo uporabljali skozi meritev, pi različnih položajih uteži in pri različnih vrtenjih. (sicer to ni potrebno, ampak nam olajša kasnejše obdelovanje podatkov in primerjavo).
Sam bom razdelil to nalogo na tri poglavja, kot so navedena v poglavju naloga. Pri vsakem delu pa je bil postopek isti (npr blizu ali na sredini krogle).
\begin{enumerate}
\item Za meritev precesije smo morali najprej postaviti kroglo na našo "zračno blazino", ki ji je pomagala, da se je vrtela s konstantno frekvenco. Nato smo, ko se je vrtavka nekaj časa vrtela in tako vzpostavila neko konstantno hitrost, prižgali stroboskop, da bi izmerili njeno hitrost vrtenja. V navodili točno piše, kako to storimo in kako nastavimo stroboskop za najboljšo meritev. Za konec smo morali le še z svinčnikom in ravnilom izmakniti vrtavko za nek kot in izmerili čas precesije. To meritev smo nato ponovili za različne hitrosti vrtenja vrtavke in različne kote nagiba vrtavke. 
\item Za meritev nutacije pa smo potrebovali vrtavko, ko se je ta vrtela, rahlo poriniti s svinčnikom in nato opazovati njeno gibanje in meriti, kolikšen je čas nutacije.
\end{enumerate}

\pagebreak
\section{Meritve in rezultati}
\subsection{Podatki o vrtavki}

V samem poročilu o vrtavki so nam navedli določene podatke, ki jih bom zaradi lažje kasnejše obdelave dodal v svoje poročilo.
\begin{table}[ht]
	\centering
	\begin{tabular}{|c|c|c|c|}
		\hline
		Predmet & masa [g] & premer [mm] & debelina ali dolžina [mm] \\
		\hline
		\hline
		Krogla & $m_k$ = 512 & $2r_k$ = 50.7 & / \\
		\hline
		Plošcica & $m_o$ = 19 & $2r_o$ = 58 & $h_o$ = 1\\
		\hline
		Palica & $m_p$ = 23 & $2r_p$ = 6.3 & $h_p$ = 89.8\\
		\hline
		Utež & $m_u$ = 19 & $2r_u$ = 19.3 & $h_u$ = 25.2 \\
		\hline
		\end{tabular}
		\caption{Podatki o naših predmetih, masa ter dolžina}
		\label{tab:FirstTable}
\end{table}

\begin{large}\textbf{Opozorilo}\end{large}\\
\medskip
Meritve smo, zaradi izrednih razmer, dobili kar od asistentov in sicer:

\subsubsection{Utež ob krogli}

\begin{table}[h]
	
	\centering
	\begin{tabular}{|c|c c|}
		
		\hline
		frekvenca vrtenja [rpm] & \multicolumn{2}{|c|}{čas precesije [s]} \\
		 & 1. kot & 2. kot \\
		\hline
		\hline
		1386 &	4,7 &	5\\
		\hline
		1415 &	5,2 &	4,9\\
		\hline
		1475 &	5,2	&   5,4\\
		\hline
		1518 &	4,8	&	5,1\\
		\hline
		1550 &	5,3	&	5\\
		\hline
		1570 &	5,2	&	5,3\\
		\hline
		
	\end{tabular}
	\caption{Utež ob krogli - precesija}
\end{table}

\begin{table}[h]
	
	\centering
	\begin{tabular}{|c|c|}
		\hline
		frekvenca vrtenja [rpm] & frekvenca nutacije [rpm]\\
		\hline
		\hline
		830	&	363\\
		\hline
		996	& 	391\\
		\hline
		1393 &	644\\
		\hline
	
		
	\end{tabular}
	\caption{Utež ob krogli - nutacija}
	
\end{table}

Ker pa so frekvence vrtenja podane v [rpm], jih moramo pretvoriti v [1/s], kar nam pomaga zveza med tem dvema količinama in sicer:

$$60 \text{rpm}=\frac{2\pi}{s}$$

Tako lahko vse naše frekvence v obratih na minuto spremenimo v kotne frekvence. \\

Da pa bi izmerili, kolikšna je hitrost precesije in nutacije, pa si pomagamo z zvezo, o frekvencah in sicer:

$$\omega_{pr,n}=\frac{2\pi}{t}$$

\pagebreak
Če sedaj z navedenimi formulami izračunamo, kolikšna je precesija in nulacija, dobimo: \\

\begin{table}[h]
	\centering
	\begin{tabular}{|c|c|c|}
		\hline
		\rule{0pt}{3ex}   
		$\omega_z$ & $\omega_{pr1}$\: & $\omega_{pr2}$\: \\
		\hline
		\hline
		145,1 &	1,3 &	1,3 \\
		\hline
		148,1 &	1,2 &	1,3 \\
		\hline
		154,4 &	1,2 &	1,2 \\
		\hline
		158,9 &	1,3 &	1,2 \\
		\hline
		162,2 &	1,2 &	1,3 \\
		\hline
		164,3 &	1,2 &	1,2 \\
		\hline
		
		
	\end{tabular}
\caption{Precesija - utež ob krogli}
\end{table}


\begin{table}[h]
	\centering
	\begin{tabular}{|c|c|}
		\hline
		\rule{0pt}{3ex}   
		$\omega_z$ & $\omega_{N}$ \\
		\hline
		\hline
		86,9 &	38,0 \\  
		\hline
		104,2 &	40,9 \\  
		\hline
		145,8 &	67,4  \\ 
		\hline
		
		
	\end{tabular}
\caption{Nutacija - utež ob krogli}
\end{table}
\pagebreak

\subsubsection{Utež 1.5cm od krogle}
Isti postopek za vse ponovimo tudi pri utežeh na različnih lokacijah, zato bom tu zgolj navedel, kakšne so vrednosti in kakšne so meritve: \\
\begin{table}[h]
	\centering
	\begin{tabular}{|c|c c|}
		\hline
		frekvenca vrtenja [rpm] & \multicolumn{2}{|c|}{čas precesije [s]} \\
		& 1. kot & 2. kot \\
		\hline
		\hline
		1073 &	3,4 &	3,6\\
		\hline
		1140 &	3,7 &	3,5\\
		\hline
		1191 &	3,7 &	3,5\\
		\hline
		1211 &	3,8 &	3,6\\
		\hline
	\end{tabular}
	\caption{Utež 1.5cm od krogle - precesija}
\end{table}

\begin{table}[h]
	\centering
	\begin{tabular}{|c|c|}
		\hline
		frekvenca vrtenja [rpm] & frekvenca nutacije [rpm]\\
		\hline
		\hline
		1220 &	631\\
		\hline
		1467 &	709\\
		\hline
		938	& 	480\\
		\hline
	\end{tabular}
	\caption{Utež 1.5cm od krogle - nutacija}
\end{table}
Za rezultate pa dobimo kar: \\

\begin{table}[h]
	\centering
	\begin{tabular}{|c|c|c|}
		\hline
		\rule{0pt}{3ex}   
		$\omega_z$ & $\omega_{pr1}$\: & $\omega_{pr2}$\\
		\hline
		\hline
		112,3 &	1,8 &	1,7 \\
		\hline
		119,3 &	1,7 &	1,8 \\
		\hline
		124,7 &	1,7 &	1,8 \\
		\hline
		126,8 &	1,7 &	1,7 \\
		\hline
	\end{tabular}
	\caption{Precesija - utež 1.5cm od krogle}
\end{table}

\begin{table}[hbt!]
	\centering
	\begin{tabular}{|c|c|}
		\hline
		\rule{0pt}{3ex}   
		$\omega_z$ & $\omega_{N}$ \\
		\hline
		\hline
		127,7 &	66,0 \\
		\hline
		153,5 &	74,2 \\
		\hline
		98,2 &	50,2 \\
		\hline
	\end{tabular}
	\caption{Nutacija - utež 1.5cm od krogle}
\end{table}


\pagebreak

\subsubsection{Utež oddaljena 2.5 cm od krogle}
Isti postopek za vse ponovimo tudi pri utežeh na različnih lokacijah, zato bom tu zgolj navedel, kakšne so vrednosti in kakšne so meritve: \\
\begin{table}[h]
	\centering
	\begin{tabular}{|c|c c|}
		\hline
		frekvenca vrtenja [rpm] & \multicolumn{2}{|c|}{čas precesije [s]} \\
		& 1. kot & 2. kot \\
		\hline
		\hline
		1100 &	2,9 &	3,1\\
		\hline
		1224 &	3,5 &	3,7\\
		\hline
		1270 &	3,7 &	3,6\\
		\hline
		1465 &	4 &		4,2\\
		\hline
	\end{tabular}
	\caption{Utež 2.5cm od krogle - precesija}
\end{table}

\begin{table}[h]
	\centering
	\begin{tabular}{|c|c|}
		\hline
		frekvenca vrtenja [rpm] & frekvenca nutacije [rpm]\\
		\hline
		\hline
		880 &	439\\
		\hline
		985 &	496\\
		\hline
		1107 &	567\\
		\hline
		
		
	\end{tabular}
	\caption{Utež 2.5cm od krogle - nutacija}
\end{table}

Za rezultate pa dobimo kar: \\

\begin{table}[h]
	\centering
	\begin{tabular}{|c|c|c|}
		\hline
		\rule{0pt}{3ex}   
		$\omega_z$ & $\omega_{pr1}$\: & $\omega_{pr2}$\\
		\hline
		\hline
		115,1 &	2,2 &	2,0 \\
		\hline
		128,1 &	1,8 &	1,7 \\
		\hline
		132,9 &	1,7 &	1,7 \\
		\hline
		153,3 &	1,6 &	1,5 \\
		\hline
		
		
	\end{tabular}
	\caption{Precesija - utež 2.5cm od krogle}
\end{table}

\begin{table}[H]
	\centering
	\begin{tabular}{|c|c|}
		\hline
		\rule{0pt}{3ex}   
		$\omega_z$ & $\omega_{N}$ \\
		\hline
		\hline
		92,1 &	45,9 \\
		\hline
		103,1 &	51,9 \\
		\hline
		115,9 &	59,3 \\
		\hline
	\end{tabular}
	\caption{Nutacija - utež 2.5cm od krogle}
\end{table}
	
S tem smo izmerili vse potrebne reči, sedaj pa potrebujemo to primerjati z dejanskimi vrednosti.
Pred tem, pa nas čaka še obravnava vztrajnostnih momentov za naše opazovano telo.

\pagebreak
\section{Izračuni}
\subsection{Izračuni vztrajnostnih momentov}
Določeni vztrajnostni momenti so nam že znani, saj smo se jih učili na Klasični fiziki in so tudi zapisani v navodilih za vajo, zato jih lahko tu zgolj navedemo za ponovitev. \\

Vztrajnostni moment krogle:

\begin{equation}
	J_{krogla} = \frac{2}{5}mr^2	
	\notag
\end{equation}

Vztrajnostni moment valja skozi os z (simetrijska os):

\begin{equation}
J_{valj,\: z} = \frac{1}{2}mr^2	
\notag
\end{equation}


Vztrajnostni moment valja skozi osi x in y (osi pravokotni na simetrijsko os):

\begin{equation}
J_{valj,\: x} = J_{valj,\: y} = m(\frac{r^2}{4} + \frac{h^2}{12})	
\notag
\end{equation}

Če gremo nato računati vztrajnostne momente naših teles vsakega posebej, preden jih sestavimo skupaj v vrtavko, dobimo: 

\begin{align*}
	& J_{krogla} = \frac{2}{5}m_kr^2_k = (1316\pm1)\: g\cdot cm^2\\
	& J_{plo\check{s}\check{c}ica,\: z} = \frac{1}{2}m_or^2_o = (80\pm7)\: g\cdot cm^2\\
	& J_{palica,\: z} = \frac{1}{2}m_pr^2_p = (1.1\pm0.1)\: g\cdot cm^2\\
	& J_{ute\check{z},\: z} = \frac{1}{2}m_ur^2_u = (9\pm0.5)\: g\cdot cm^2\\
	& J_{plo\check{s}\check{c}ica,\: x} = m_o(\frac{r^2_o}{4} + \frac{h^2_o}{12}) = (42\pm0.3)\: g\cdot cm^2\\\
	& J_{palica,\: x} = m_p(\frac{r^2_p}{4} + \frac{h^2_p}{12}) = (160\pm7)\: g\cdot cm^2\\
	& J_{ute\check{z},\: x} = m_u(\frac{r^2_u}{4} + \frac{h^2_u}{12}) = (14\pm0.8)\: g\cdot cm^2\		
\end{align*}

Ker bomo po naši palici sedaj premikali utež, nas zanima, kako se bo z premikanjem naše uteži spreminjalo težišče. Za izračun težišča si lahko pomagamo s formulo.

$$l = \frac{\sum_{i}m_i\cdot h_i}{\sum_{i}m_i}$$

Kjer so i indeksi vseh teles, ki nastopajo v našem sistemu. Za vrtavko bo veljalo, da je: 

	$$l =  \frac{m_o\cdot \left(r_k + \frac{h_o}{2}\right) + m_p\cdot \left(r_k + h_o + \frac{h_p}{2}\right) + m_u\cdot L}{m_k + m_o + m_p + m_u}\\
	=  0.3717cm + 0.033\cdot L$$

Kjer je L drsnik, ki nam pove, kako nizko ali visoko smo postavili utež, oziroma oddaljenost uteži od središča krogle. Ta bo za vsako lokacijo drugačen. 

\pagebreak
Za našo nalogo izračuna hitrosti precesije in nulacije nam bosta koristili formuli iz uvoda in sicer: 

$$
\omega_{\mathrm{N}}=\frac{J_{33}}{J_{11}} \omega_{z} \qquad \omega_{\mathrm{pr}}=\frac{m g l}{J_{33} \omega_{z}}
$$

Posledično vidimo, da bomo potrebovali izračunati $J_{33}$ in $J_{11}$. Kaj pa sploh sta te dva vztrajnostna momenta?\\
To sta vztrajnostna momenta, ki sta odvisna od tega, okoli katere osi se vrtavka vrti. Saj vrtavko lahko vrtimo na različne načine, okoli x, y ali z osi. \\
\medskip

Tako lahko za vsako os izračunamo, kolikšen je, ampak v našem primeru potrebujemo zgolj $J_{33}$ in $J_{11}$.

Pred izračunom pa bom definiral še naslednje pojme (kakšnega smo spoznali že prej, a jih bom zaradi preglednosti ponovno napisal):
\begin{enumerate}
\item M - skupna masa
\item l - oddaljenost težišča od središča krogle
\item L - oddaljenost uteži od središča krogle
\end{enumerate}

Sedaj pa izračunajmo oba vztrajnostna momenta:
\begin{enumerate}

\item Vztrajnostni moment $J_{11}$: 
\begin{center}
	$J_{11} =  J_{krogla} + J_{plo\check{s}\check{c}ica,\: x} + J_{palica,\: x} + J_{ute\check{z},\: x} + m_o\cdot \left(r_k+\frac{h_o}{2}\right)^2 + m_p\cdot \left(r_k + h_o + \frac{h_p}{2}\right)^2 + m_u\cdot L^2$ \\
	\medskip
	Rezultat tega izračuna je: \underline{$J_{11}=2727g\cdot cm^2 + 19g\cdot L^2$}
\end{center}

\item Vztrajnostni moment $J_{33}$:
\begin{center}
$J_{33} = J_{krogla} + J_{plo\check{s}\check{c}ica,\: z} + J_{palica,\: z} + J_{ute\check{z},\: z}$\\
\medskip
Rezultat tega izračuna je: \underline{$J_{33}=1406 g\cdot cm^2$}
\end{center}
\end{enumerate}

S temi vsemi podatki pa lahko izračunamo tudi predvidene kotne hitrosti za naše dva pojava.

\subsection{Izračuni predvidenih kotnih hitrosti}
Za to lahko uporabimo že prej navedene formule, ki so: 
$$
\omega_{\mathrm{N}}=\frac{J_{33}}{J_{11}} \omega_{z} \qquad \omega_{\mathrm{pr}}=\frac{m g l}{J_{33} \omega_{z}}
$$
In tako izračunamo vrednosti in jih primerjamo z izmerjenimi. \\

\pagebreak
\subsubsection{Utež ob krogli}
Potrebne manjkajoče podatke izračunamo z $L = r_k + h_0 + \frac{h_u}{2} = 3.895 cm$.

\begin{align*}
	& J_{11} = 3015g\cdot cm^2\\ 
	& l = 0.50 cm\\
	& mgl = 280\: 900\frac{g\cdot cm^2}{s^2}
\end{align*}

\begin{table}[h]
	\centering
	\begin{tabular}{|c|c|c|c|}
		\hline
		\rule{0pt}{3ex}   
		$\omega_z$ & $\omega_{pr1}$\: & $\omega_{pr2}$\: & $\omega_{pr,calc}$\: (izračunana)\\
		\hline
		\hline
		145,1 &	1,3 &	1,3 &	1,38\\
		\hline
		148,1 &	1,2 &	1,3 &	1,35\\
		\hline
		154,4 &	1,2 &	1,2 &	1,29\\
		\hline
		158,9 &	1,3 &	1,2 &	1,26\\
		\hline
		162,2 &	1,2 &	1,3 &	1,23\\
		\hline
		164,3 &	1,2 &	1,2 &	1,22\\
		\hline
		
		
	\end{tabular}
\caption{Precesija - utež ob krogli}
\end{table}

\begin{table}[h]
	\centering
	\begin{tabular}{|c|c|c|}
		\hline
		\rule{0pt}{3ex}   
		$\omega_z$ & $\omega_{N}$ & $\omega_{N,calc}$\: (izračunana)\\
		\hline
		\hline
		86,9 &	38,0 &	 40,5\\  
		\hline
		104,2 &	40,9 &	 48,6\\  
		\hline
		145,8 &	67,4 &	 68,0 \\ 
		\hline
		
		
	\end{tabular}
\caption{Nutacija - utež ob krogli}
\end{table}
Komentar tabele: \\
\begin{enumerate}
\item Precesija: \\
V tabeli za precesijo imamo v prvem stolpcu navedeno  kotno hitrost vrtavke, ki smo jo izmerili, drugi in tretji stolpec nam prikazujeta kotne hitrosti precesije, ki smo jih izračunali v poglajvu meritve (postopek podatkov je zapisan tam), zadnji stolpec pa nam prikazuje izračunane vrednosti iz naših vztrajnostnih momentov v odvisnosti od začetne kotne hitrosti vrtavke $\omega_z$.
\item Nutacija: \\
V tabeli za nutacijo imamo v prvem stolpcu navedeno kotno hitrost vrtavke, ki smo jo izmerili, drugi stolpec nam prikazuje kotno hitrost nutacije, ki smo jo izračunali v poglavju meritve (postopek podatkov je zapisan tam),zadnji stolpec pa nam prikazuje izračunane vrednosti iz naših vztrajnostnih momentov v odvisnosti od začetne kotne hitrosti vrtavke $\omega_z$.
\end{enumerate}


\pagebreak
\subsubsection{Utež 1.5cm oddaljena od krogle}

\begin{align*}
& J_{11} = 3259g\cdot cm^2\\ 
& l = 0.55 cm\\
& mgl = 306\: 800\frac{g\cdot cm^2}{s^2}
\end{align*}

\begin{table}[h]
	\centering
	\begin{tabular}{|c|c|c|c|}
		\hline
		\rule{0pt}{3ex}   
		$\omega_z$ & $\omega_{pr1}$\: & $\omega_{pr2}$\: & $\omega_{pr,calc}$\: (izračunana)\\
		\hline
		\hline
		112,3 &	1,8 &	1,7 &	1,94\\
		\hline
		119,3 &	1,7 &	1,8 &	1,83\\
		\hline
		124,7 &	1,7 &	1,8 &	1,75\\
		\hline
		126,8 &	1,7 &	1,7 &	1,72\\
		\hline
		
		
	\end{tabular}
	\caption{Precesija - utež 1.5cm od krogle}
\end{table}

\begin{table}[h]
	\centering
	\begin{tabular}{|c|c|c|}
		\hline
		\rule{0pt}{3ex}   
		$\omega_z$ & $\omega_{N}$ & $\omega_{N,calc}$\: (izračunana)\\
		\hline
		\hline
		127,7 &	66,0 &	55,1\\
		\hline
		153,5 &	74,2 &	66,2\\
		\hline
		98,2 &	50,2 &	42,4\\
		\hline
		
		
	\end{tabular}
	\caption{Nutacija - utež 1.5cm od krogle}
\end{table}

\vspace{2mm}
Komentar: podatek oddaljenosti 1.5cm, sem privzel, da je to razdalja med sredino uteži in kroglo.\\

Komentar tabele: 
\begin{enumerate}
\item Precesija: \\
V tabeli za precesijo imamo v prvem stolpcu navedeno  kotno hitrost vrtavke, ki smo jo izmerili, drugi in tretji stolpec nam prikazujeta kotne hitrosti precesije, ki smo jih izračunali v poglajvu meritve (postopek podatkov je zapisan tam), zadnji stolpec pa nam prikazuje izračunane vrednosti iz naših vztrajnostnih momentov v odvisnosti od začetne kotne hitrosti vrtavke $\omega_z$.
\item Nutacija: \\
V tabeli za nutacijo imamo v prvem stolpcu navedeno kotno hitrost vrtavke, ki smo jo izmerili, drugi stolpec nam prikazuje kotno hitrost nutacije, ki smo jo izračunali v poglavju meritve (postopek podatkov je zapisan tam),zadnji stolpec pa nam prikazuje izračunane vrednosti iz naših vztrajnostnih momentov v odvisnosti od začetne kotne hitrosti vrtavke $\omega_z$.
\end{enumerate}

\newpage
\subsubsection{Utež 2.5cm oddaljena od krogle}

\begin{align*}
& J_{11} = 3479g\cdot cm^2\\ 
& l = 0.58 cm\\
& mgl = 325\: 400\frac{g\cdot cm^2}{s^2}
\end{align*}

\begin{table}[h]
	\centering
	\begin{tabular}{|c|c|c|c|}
		\hline
		\rule{0pt}{3ex}   
		$\omega_z$ & $\omega_{pr1}$\: & $\omega_{pr2}$\: & $\omega_{pr,calc}$\: (izračunana)\\
		\hline
		\hline
		115,1 &	2,2 &	2,0 &	2,01\\
		\hline
		128,1 &	1,8 &	1,7 &	1,81\\
		\hline
		132,9 &	1,7 &	1,7 &	1,74\\
		\hline
		153,3 &	1,6 &	1,5 &	1,51\\
		\hline
		
		
	\end{tabular}
	\caption{Precesija - utež 2.5cm od krogle}
\end{table}

\begin{table}[h]
	\centering
	\begin{tabular}{|c|c|c|}
		\hline
		\rule{0pt}{3ex}   
		$\omega_z$ & $\omega_{N}$ & $\omega_{N,calc}$\: (izračunana)\\
		\hline
		\hline
		92,1 &	45,9 &	37,2\\
		\hline
		103,1 &	51,9 &	41,7\\
		\hline
		115,9 &	59,3 &	46,8\\
		\hline
		
		
	\end{tabular}
	\caption{Nutacija - utež 2.5cm od krogle}
\end{table}
\vspace{2mm}
Komentar: podatek oddaljenosti 2.5cm, sem privzel, da je to razdalja med sredino uteži in kroglo.\\

Komentar tabele:
\begin{enumerate}
\item Precesija: \\
V tabeli za precesijo imamo v prvem stolpcu navedeno  kotno hitrost vrtavke, ki smo jo izmerili, drugi in tretji stolpec nam prikazujeta kotne hitrosti precesije, ki smo jih izračunali v poglajvu meritve (postopek podatkov je zapisan tam), zadnji stolpec pa nam prikazuje izračunane vrednosti iz naših vztrajnostnih momentov v odvisnosti od začetne kotne hitrosti vrtavke $\omega_z$.
\item Nutacija: \\
V tabeli za nutacijo imamo v prvem stolpcu navedeno kotno hitrost vrtavke, ki smo jo izmerili, drugi stolpec nam prikazuje kotno hitrost nutacije, ki smo jo izračunali v poglavju meritve (postopek podatkov je zapisan tam),zadnji stolpec pa nam prikazuje izračunane vrednosti iz naših vztrajnostnih momentov v odvisnosti od začetne kotne hitrosti vrtavke $\omega_z$.
\end{enumerate}

\pagebreak
\section{Napaka}
Pri primerjavi izračunov in meritev vidimo, da pride do določenih odstopanj. Sam sklepam, da se napaka predvsem pojavi zaradi približka naše hitre vrtavke, ki smo ga ustvarili v enačbi: 
\begin{equation}
	\label{skrajna kota}
	\cos\theta_0 - \cos\theta_1 \approx \frac{J_{11}}{J_{33}}\frac{2mgl}{J_{33}\omega^2_z}\sin^2\theta_0
\end{equation}
Saj vidimo, da bodo za manjše hitrosti vrtenja, večja odstopanja od naše meritve. To se dobro vidi pri večini meritve precesije, saj se z večanjem hitrosti vrtenja vrtavke manjša odstopanje meritve od naše dejanske vrednosti. Primer:
\begin{table}[h]
	\centering
	\begin{tabular}{|c|c|c|c|}
		\hline
		\rule{0pt}{3ex}   
		$\omega_z$ & $\omega_{pr1}$\: & $\omega_{pr2}$\: & $\omega_{pr,calc}$\: (izračunana)\\
		\hline
		\hline
		145,1 &	1,3 &	1,3 &	1,38\\
		\hline
		148,1 &	1,2 &	1,3 &	1,35\\
		\hline
		154,4 &	1,2 &	1,2 &	1,29\\
		\hline
		158,9 &	1,3 &	1,2 &	1,26\\
		\hline
		162,2 &	1,2 &	1,3 &	1,23\\
		\hline
		164,3 &	1,2 &	1,2 &	1,22\\
		\hline
		
		
	\end{tabular}
\caption{Precesija - utež ob krogli, primer da se napaka z večanjem hitrosti vrtenja, manjša}
\end{table}
Na žalost isti princip ne moremo reči za nutacijo, kjer pa ta približek ni tako očiten. Je pa rahlo viden. Predvsem iz tega razloga, da z oddaljenostjo uteži od središča krogle, se napaka veča, oziroma so odstopanja večja. Kot primer lahko vzamemo nutacije za največje hitrosti pri vseh odmikih in jih damo v tabelo.
\begin{table}[h]
	\centering
	\begin{tabular}{|c|c|c|c|}
		\hline
		\rule{0pt}{3ex}   
		Odmik uteži & 0 & 1.5 & 2.5\\
		\hline
		\hline
		$\omega_z$ & 145,8  & 153,5 & 115,9\\
		\hline
		$\omega_{N}$ & 67,4 & 74,2 & 59,3  \\
		\hline
		$\omega_{N,calc}$\: (izračunana) & 68,0 & 66,2 & 46,8 \\
		\hline
		$\Delta \omega_N$ & -0,6 & 8 & 13,5 \\
		\hline
	\end{tabular}
	\caption{Nutacija, primerjava vrednosti za različne odmike.}
\end{table}
\vspace{2mm}
	
\end{document}


