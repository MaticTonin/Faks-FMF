\documentclass[12pt,landscape]{article}
\usepackage{multicol}
\usepackage{calc}
\usepackage{ifthen}
\usepackage[landscape]{geometry}
\usepackage{amsmath,amsthm,amsfonts,amssymb}
\usepackage{color,graphicx,overpic}
\usepackage{hyperref}
\usepackage{amsthm}
\usepackage{mathrsfs}
\usepackage{enumerate}
\usepackage{enumitem}
\usepackage{icomma}

\pdfinfo{
  /Title (Uporabne formule iz klasične mehanike za fizike)
  /Author (Urban Duh)
  /Subject (Klasična mehanika)}

% This sets page margins to .5 inch if using letter paper, and to 1 cm
% if using A4 paper. (This probably isn't strictly necessary.)
% If using another size paper, use default 1cm margins.
\ifthenelse{\lengthtest { \paperwidth = 11in}}
    { \geometry{top=.5in,left=.5in,right=.5in,bottom=.5in} }
    {\ifthenelse{ \lengthtest{ \paperwidth = 297mm}}
        {\geometry{top=1cm,left=1cm,right=1cm,bottom=1cm} }
        {\geometry{top=1cm,left=1cm,right=1cm,bottom=1cm} }
    }

% Turn off header and footer
\pagestyle{empty}

% Redefine section commands to use less space
\makeatletter
\renewcommand{\section}{\@startsection{section}{1}{0mm}%
                                {-1ex plus -.5ex minus -.2ex}%
                                {0.5ex plus .2ex}%x
                                {\normalfont\large\bfseries}}
\renewcommand{\subsection}{\@startsection{subsection}{2}{0mm}%
                                {-1explus -.5ex minus -.2ex}%
                                {0.5ex plus .2ex}%
                                {\normalfont\normalsize\bfseries}}
\renewcommand{\subsubsection}{\@startsection{subsubsection}{3}{0mm}%
                                {-1ex plus -.5ex minus -.2ex}%
                                {1ex plus .2ex}%
                                {\normalfont\small\bfseries}}
\makeatother

% Don't print section numbers
\setcounter{secnumdepth}{0}

\setlength{\parindent}{0pt}
\setlength{\parskip}{0pt plus 0.5ex}

%My Environments
\newcommand{\norm}[1]{\left\lVert#1\right\rVert}
\newcommand{\avg}[1]{\langle#1\rangle}
\newcommand{\R}{\mathbb{R}}
\newcommand{\rv}{\vec{r}}
\newcommand{\vv}{\vec{v}}
\newcommand{\av}{\vec{a}}
\newcommand{\F}{\vec{F}}
\newcommand{\uunderline}[1]{\underline{\underline{#1}}}

% -----------------------------------------------------------------------

\begin{document}
\raggedright
\footnotesize
\begin{multicols}{3}
% multicol parameters
% These lengths are set only within the two main columns
%\setlength{\columnseprule}{0.25pt}
\setlength{\premulticols}{1pt}
\setlength{\postmulticols}{1pt}
\setlength{\multicolsep}{1pt}
\setlength{\columnsep}{2pt}

\section{Osnovni pojmi}
Splošen zapis za katerekoli koordinate: \\
$\vec e_\alpha = \frac{\partial \rv}{\partial x_\alpha}$ \\
$d \rv = \sum \frac{\partial \rv}{\partial x_\alpha} d x_\alpha$ \\
$\vv = \dot \rv = \sum \dot x_\alpha \vec e_\alpha$  \\
Bazni vektorji: \\
$\hat{i}=\frac{\partial \vec{r}}{\partial x}$
$\hat{j}=\frac{\partial \vec{r}}{\partial y}$
$\hat{k}=\frac{\partial \vec{r}}{\partial z}$ \medskip \\
\textbf{Polarne koordinate za en delec:} \\ 
$\rv = r \cos \varphi \hat i + r \sin \varphi \hat j$ \\
$\vec e_r = \cos \varphi \hat i + \sin \varphi \hat j$ \\
$\vec e_\varphi = - r \sin \varphi \hat i + r \cos \varphi \hat j$ \\
$|\vec e_r| = 1$ \qquad $|\vec e_\varphi| = r$ \qquad $\vec e_r \cdot \vec e_\varphi = 0$ \\
$T= \frac{1}{2} m (\dot{r}^2 + r^2 \dot{\varphi}^2)$ \\
$\av = \vec e_r \left(\ddot r - \dot{\varphi}^2 r\right) + \vec e_\varphi \left(\ddot \varphi + 2 \frac{\dot \varphi \dot r}{r}\right)$  \\
\textbf{Pozor:} \\
Radij se lahko spreminja in je funkcija nečesa drugega, primer, točka, v valju. Takrat raje zapiši navadne koordinate.\medskip \\
\textbf{Sferične koordinate} \\
$\rv = r \cos \varphi \sin \vartheta \hat i + r \sin \varphi \sin \vartheta \hat j + r \cos \vartheta \hat k$ \\
$\vec e_r =  \cos \varphi \sin \vartheta \hat i + \sin \varphi \sin \vartheta \hat j + \cos \vartheta \hat k$ \\
$\vec e_\vartheta =  r \cos \varphi \cos \vartheta \hat i + r \sin \varphi \cos \vartheta \hat j - r \sin \vartheta \hat k$ \\
$\vec e_\varphi =  -r \sin \varphi \sin \vartheta \hat i + r \cos \varphi \sin \vartheta \hat j$ \\
$|\vec e_r| = 1$ \qquad $|\vec e_\vartheta| = r$ \qquad $|\vec e_\varphi| = r \sin \vartheta$ \qquad $\vec e_\alpha \cdot \vec e_\beta = 0$ \\
$T = \frac{1}{2} m (\dot{r}^2 + \dot{\varphi}^2 r^2 \sin^2 \vartheta + \dot{\vartheta}^2 r^2)$ \medskip \\

\textbf{Reševanje nalog:}\\
V primeru, da naloga ni mišljena za reševanje z Lagrangevim formalizmom, zapiši sile v vseh vektorskih smereh z njihovimi baznimi vektorji. Oz zapiši 2. NZ za sistem. \medskip \\
\subsection{Neinercialni koordinatni sistemi}
Lasten sistem je označen z črticamo, medtem ko je celoten sistem označen brez črtic. \medskip \\
$\frac{d \vec e_\alpha'}{dt} = \vec \omega' \times \vec e_\alpha'$ \\
$\rv' = \sum x_\alpha' \vec e_\alpha' = \rv - \vec R$ \\
Relativna hitrost, ki deluje na delec. v celotnem sistemu. \\
$\vv_\mathrm{rel} = \sum \dot x_\alpha' \vec e_\alpha'$ \\
Hitrost, ki ga čuti delec v lastnem sistemu. \\
$\dot \rv' = \vv_\mathrm{rel} + \vec \omega ' \times \rv'$ \\
Pospešek, ki ga čuti delec v lastnem sistemu. \\
$\ddot \rv' = \av_\mathrm{rel} + 2 \vec \omega' \times \vv_\mathrm{rel} + \vec \omega' \times (\vec \omega' \times \rv') + \dot{\vec \omega}' \times \rv'$ \\
Sistemska sila, ki jo občuti delec: \\
$\F_s = - m \ddot{\vec R} - m \vv_\mathrm{rel} - 2 m \vec \omega' \times \vv_\mathrm{rel} - m \vec \omega' \times (\vec \omega' \times \rv') - m \dot{\vec \omega}' \times \rv'$ \\
$m \av_\mathrm{rel} = m \sum \ddot x_\alpha \vec e_\alpha = \sum \F + \F_s$ \medskip \\

\textbf{Vrteči sistem:} \\
Koordinate telesa v lastnem sistemu so:
$\hat i' = \cos(\Omega t) \hat i + \sin(\Omega t) \hat j$ \\
$\hat j' = - \sin(\Omega t) \sin \varphi \hat i + \cos(\Omega t) \sin \varphi \hat j + \cos \varphi \hat k$ \\
$\hat k' = \sin(\Omega t) \cos \varphi \hat i - \cos(\Omega t) \cos \varphi \hat j + \sin \varphi \hat k$ \\
Kjer je omega hitrost vrtenja celotnega sistema. \medskip \\
$\vec \Omega = \Omega \cos \varphi \hat j' + \Omega \sin \varphi \hat k'$ \\
$\vec R = -R \hat k'$ \qquad $\dot{\vec R} = \vec \Omega \times \vec R$ \\
$\ddot \rv = \dot{\vv}_\mathrm{rel} + \vec \Omega \times \left( \dot \rv' + \dot{\vec R} \right) = \av_\mathrm{rel} + 2 \vec \Omega \times \vv_\mathrm{rel} + \vec \Omega \times \vec \Omega \times \left( \rv' + \vec R \right)$ \\
$m\av_\mathrm{rel} = \sum \F - 2 m\vec \Omega \times \vv_\mathrm{rel} - m \vec \Omega \times \vec \Omega \times \left( \rv' + \vec R \right)$ \medskip \\
Uporabna substitucija pri reševanju takih nalog, kjer dobimo gibanje v x in y smeri oz dve diferencialni enačbi. Seštejemo ju in uporabimo substitucijo:
$\zeta=x+iy$ \medskip \\
V primeru, da je sistem nagnjen, torej da $\hat{k} \neq \hat{k}'$ z Eulerjevo rotacijo najdi smer med $\omega$ in $\omega'$.
\subsection{Pomembne količine}

\textbf{1 delec} \\

$\F = m \ddot \rv$ \\
$\F_{21} = - \F_{12}$ \smallskip \\

$\vec p = m \vec v$ \\
$\frac{d \vec p}{dt} = \F$ \smallskip \\

$\vec l = \rv \times \vec p = m\rv \times \vv$ \\
$\frac{d \vec l}{dt} = \vec M = \rv \times \F$ \smallskip \\

$A = \int_\gamma \F \cdot d\rv$ \\
$T = \frac{1}{2} m \vv \cdot \vv$ \smallskip \\

Sila je konzervativna, če $\F = - \nabla U$. \\
$\F$ konzervativna $\iff$ $\frac{\partial F_x}{\partial y} = \frac{\partial F_y}{\partial x}$ (v 3D po parih $xy$, $yz$, $xz$). \\ 
$\F$ konzervativna $\iff$ $A = \int_\gamma \F \cdot \rv = - \Delta U$ neodvisno od poti. \medskip \\

\textbf{Sistem $\mathbf{N}$ delcev} \\
$\rv_T = \frac{\sum m_i \rv_i}{\sum m_i}$ \\
$\vec P = \sum \vec p_i = M \rv_T$ \\
$\frac{d \vec P}{dt} = M \av_T = \F^{(z)}$ \smallskip \\

$\vec L = \sum \vec l_i = \rv_T \times M \vv_T + \sum \rv_i' \times m \vv_i'$ \\
$\dot{\vec L} = \vec M^{(z)} + \vec M^{(n)} = \sum \rv_i \times \F_i^{(z)} + \sum_{j < i} (\rv_i - \rv_j) \times \F_{ij}$ \\
Če so sile centralne $\left( \F_{ij} = F_{ij} \frac{\rv_i - \rv_j}{|\rv_i - \rv_j|} \right)$, je $\vec M^{(n)} = 0$.\smallskip \\

$A = A^{(z)} + A^{(n)}$ \\
$T = \frac{1}{2} M v_T^2 + \sum \frac{1}{2} m_i v_i'^2$ \\
$V = \sum_i V_i + \frac{1}{2} \sum_{ij, i \neq j} V_{ij}$\medskip \\

\textbf{Virialni teorem} (sistem $N$ delcev) \\
$2 \avg T = \frac{1}{2} \avg{\sum_{ij, i \neq j} \frac{\partial V_{ij}}{\partial r_{ij}} r_{ij}} - \avg{\sum_i \F_i^{z} \cdot \rv_i}$ \\
$V(\rv) = \alpha r^n$ \quad $\implies$ \quad $2 \avg{T} = n \avg{V}$ \medskip \\
\section{Vezi}
Vezi so zveze med različnimi koordinatami. Npr, da ima naš sistem preveč komponent, ki bi jih lahko vključili, zapišemo vezi, oziroma zveze med dvema koordinatama, da si Lagrangevo funkcijo poenostavimo. 
\textbf{Generalizirane koordinate:}\\
So koordinate, ki smo jih izrazili iz vezi in bomo nato gledali njihovo odvisnost v E-L enačbi.

\section{Lagrangev formalizem}

$\sum_{i = 1}^{N} \left(\vec F_i^{(a)} - m_i \vec a_i\right) \delta \vec r_i = 0$ \qquad (D'Alembertov princip) \\
V primeru, da kakšna sila ni potencialna, uvedemo:
$Q_j = \sum_{i = 1}^{N} \vec F_i^{(a)} \frac{\partial \vec r_i}{\partial q_j}$ \\
$\frac{d}{dt} \frac{\partial T}{\partial \dot{q_j}} - \frac{\partial T}{\partial q_j} = Q_j$, \quad $j = 1, 2 \dots N$ \medskip \\
Če so vse sile potencialne, torej, 
Če $\vec F_i = - \nabla_i V$, $\forall i$. Dobimo Lagrangev formalizem: \\
$L = T - V$ \\
In rešujemo sistem enačbe za vsako koordinato posebej:
$\frac{d}{dt} \frac{\partial L}{\partial \dot{q_j}} - \frac{\partial L}{\partial q_j} = 0$ \medskip \\

Če $\frac{\partial L}{\partial q_i}$=0, potem $\frac{\partial L}{\partial \dot{q_i}} = konst. = p_i$ (posplošeni impulz, $q_i$ pravimo ciklična koordinata). \medskip \\
Konstanta gibanja: \\
Če $\frac{\partial L}{\partial t} = 0$, potem $H = \sum_{j}\dot q_j \frac{\partial L}{\partial \dot q_j} - L = konst.$. \\
Če $V \neq V(\dot q_1, \dots , \dot q_n, t)$, potem $H = T + V$. \medskip \\

Lagrangeva funkcija je nedoločena do $\frac{d}{dt}F(q_1, \dots, q_n, t)$. \medskip \\

\subsection{Izrek Emmy Noether}

Imejmo enoparametrično preslikavo koordinat $q_i \mapsto Q_i(s, t)$. \\
Ta ustreza zvezni simetriji  $L$, če $\frac{\partial L(Q_i(s,t), \dot Q_i(s, t), t)}{\partial s} = 0$. \\
Tedaj obstaja ohranjena količina $\sum_i \frac{\partial L}{\partial q_i} \frac{\partial Q_i}{\partial s}|_{s = 0}$. \medskip \\

\subsection{Hamiltonov princip}

$S(L) = \int_{t_1}^{t_2}L(q_1, \dots, q_n, \dot q_1, \dots, \dot q_n, t) \, dt$ \quad (akcija) \\
Enačbe gibanja ustrezajo minimumu funkcionala akcije $S$. \\ \medskip 


\section{Problem dveh teles}

Imamo $m_1, m_2, \rv_1, \rv_2$. \\
$M \rv_t = (m_1 + m_2) \rv_t = m_1 \rv_1 + m_2 \rv_2$ \\
$m = \frac{m_1 m_2}{m_1 + m_2}$  \quad $\vec{r}_t=\frac{1}{m_1+m_2}(m_1\vec{r}_1+m_2\vec{r}_2$\\
$\rv = \rv_2 - \rv_1$ \quad $\implies$ \quad $\rv_i = \rv_t - \frac{m_i}{M} \rv$ \medskip \\

$L = \frac{1}{2} M v_t^2 + \frac{1}{2} m |\dot \rv|^2 - V(\rv, \dot \rv)$ \\
$\vec p_t = M \vv_t = konst.$ \\

\subsection{Centralni potencial}

Imamo $V = V(r)$. Za katerega velja, da je:
$\F = - \nabla V = -\frac{dV}{dr} \frac{\rv}{r}$ \\
Gibalna količina je konstantna. 
$\vec L = konst.$ \\
$\vec L \cdot \rv = 0$ \quad $\implies$ \quad Gibanje je ravninsko. \medskip \\

Definirajmo $\vec L = L \hat e_z$. (gibalna količina) \\
Lagrangeva funkcija je enaka:$L = \frac{1}{2} m (\dot r^2 + r^2 \dot \varphi^2) - V(r)$ \\
Ker je odvisnost zgolj od r, je:
$p_\varphi = m r^2 \dot \varphi = l = konst.$ \\
$\frac{dS}{dt} = \frac{1}{2}r^2 \dot \varphi = konst.$ (2. Keplerjev zakon) \medskip \\
Definirajmo efektivni potencial kot:
$V_\mathrm{ef} = \frac{p_\varphi^2}{2mr^2} + V(r)$ \\
Celotna energija takega sistema je:
$H = \frac{1}{2} m \dot r^2 + V_\mathrm{ef}$ \\
In ker je kinetični del energije večji kot nič, sledi:
$\frac{1}{2} m \dot r^2 \geq 0 \implies H \geq V_\mathrm{ef}$ $\implies$ kvalitativna določitev orbit. \medskip \\

Pogosto uporabna substitucija $u = \frac{1}{r}$. \\
Pogosto uporabno $\frac{d}{dt} = \frac{d \varphi}{dt} \frac{d}{d \varphi} = \frac{p_\varphi}{mr^2} \frac{d}{d\varphi}$ \medskip \\

\textbf{Reševanje nalog s potenciali:}\\
Če rešujemo nalogo s potenciali, najprej zapišemo celotno energojo kot $H=T+V_{\mathrm{eff}}$, kjer je $V_\mathrm{eff}=\frac{p_{phi}^2}{2mr^2}+V(r)$. In nato rešujemo problem za različne množice rešitev. Ne smemo pozabiti, da je: $\frac{d}{dt} = \frac{d \varphi}{dt} \frac{d}{d \varphi} = \frac{p_\varphi}{mr^2} \frac{d}{d\varphi}$. 
\subsection{Keplerjev problem}

Imamo $V(r) = -\frac{k}{r}$. \\
$\vec A = \vec p \times \vec L - mk \frac{\rv}{r} = konst.$ (Laplace-Runge-Lenzov vektor) \\
$\vec A \cdot \vec L = 0$ \\
$r = \frac{l^2}{mk\left(1 + \frac{A}{mk} \cos \varphi\right)} = \frac{r_0}{1 + \varepsilon \cos \varphi}$ \\
$\varepsilon = \frac{A}{mk} = \sqrt{1 + \frac{2H l^2}{k^2 m}}$\medskip \\

$\varepsilon = 0$ - krožnica - $H = - \frac{mk^2}{2l^2}$, \qquad $\varepsilon < 1$ - elipsa - $H < 0$ \\
$\varepsilon = 1$ - parabola - $H = 0$, \qquad  \qquad $\varepsilon > 1$ - hiperbola - $H > 0$ \medskip \\

Uporabna substitucija, v primeru, da je potencial $r^{n}$ ko je $n \in \mathbb{N}$ \quad $\frac{dr}{dt}=r'\dot{\varphi}$.

\subsection{Sipanje delcev}
Sipalni tok, ki bo prišel iz neke razdalje r do tarče. Curek delcev ima vhodni parameter b.\\
$dI_\mathrm{vh} = j \, dS = j 2\pi b \, db$ \\
Sipalni tok, ki se bo sipal na tarči. \\
$dI_\mathrm{iz} = j \sigma(\Omega) \, d \Omega$ \\
Sipalni presek v odvisnosti od začetega parametra in kota sipanja, pod katerim odleti delec stran od tarče. \\
$\sigma(\vartheta) = \frac{b}{\sin \vartheta} |\frac{db}{d \vartheta}|$ \qquad (diferencialni sipalni presek) \\
Pri taki nalogi moramo ponavadi izraziti b kot funkcijo$\theta$ in jo nato odvajati po $\theta$.\\
Totalni sipalni presek pa definiramo kot:
$\sigma_\mathrm{tot} = \int \sigma(\Omega) \, d \Omega = \pi b_\mathrm{max}^2$ \medskip \\
Kjer je $b_{\mathrm{max}}$ velikost tarče.

\section{Togo telo}

$|\rv_i - \rv_j| = r_{ij} = konst.$ \medskip \\

\subsection{Eulerjevi koti}
Če se telo vrti okoli določene lege, lahko to opišem z Eulerjevimi koti:  \medskip
\\


Zasuk okoli $\hat{k}$: \medskip\\


$T(\varphi) = \begin{bmatrix}
\cos \varphi & \sin \varphi & 0 \\
-\sin \varphi & \cos \varphi & 0 \\
0 & 0 & 1
\end{bmatrix}$ \qquad (precesija, $\vec r \to \vec r''$) \\
\medskip
Zasuk okoli $\hat{i''}$: \medskip\\
$U(\vartheta) = \begin{bmatrix}
1 & 0 & 0 \\
0 & \cos \vartheta & \sin \vartheta \\
0 & -\sin \vartheta & \cos \vartheta \\
\end{bmatrix}$ \qquad (nutacija, $\vec r'' \to \vec r'''$) \medskip\\


Zasuk okoli $\hat{k''}$:\medskip \\
$V(\psi) = \begin{bmatrix}
\cos \psi & \sin \psi & 0 \\
-\sin \psi & \cos \psi & 0 \\
0 & 0 & 1
\end{bmatrix}$ \qquad (rotacija, $\vec r''' \to \vec r'$)\medskip \\


$R = V(\psi) U(\vartheta) T(\varphi)$ \qquad (pasivna rotacija) \medskip \\

Za vsako rotacijo $R$ $\exists \vec n \neq 0: R \vec n = \vec n$, torej obstaja os rotacije. \medskip \\

V lastnem sistemu $S'$: \\
$\vec \omega = \dot \varphi \vec e_3 + \dot \vartheta \vec e_1'' + \dot \psi \vec e_3' = (\sin \vartheta \sin \psi \, \dot\varphi + \cos \psi \, \dot \vartheta) \vec e_1' + (\sin \vartheta \cos \psi \, \dot \varphi - \sin \psi \, \dot \vartheta) \vec e_2' + (\cos \vartheta \, \dot \varphi + \dot \psi) \vec e_3'$ \medskip \\
Če npr, če se stožec kotali: $\vec{V_p}=0=\dot{\phi}\times \vec{R}+ \dot{\psi}\times \vec{r}$
\subsection{Kinetična energija iz Eulerjevih kotov}
$$T=\frac{1}{2}(J_x \omega_x^2+J_y \omega_y^2 +J_z \omega_z^2)$$
Če je vrtavka osnosimetrična velja (J,J,J'):
$$T=\frac{1}{2}J(\dot{\phi}^2 \sin (\vartheta)+\dot{\vartheta}^2)+\frac{1}{2}J'(\dot{\phi}\cos (\vartheta)+\dot{\psi})^2$$
In še:
$$2J=J'+\int 2z^2 dm$$ \medskip \\
\subsection{Prosta precesija}
Zanjo ne potrebujemo računati Eulerjevih enačb ampak jo lahko dobimi iz Eulerjevih kotov. Iz enačbe za $\vec{\omega}$ lahko zapišemo, da je:
$\vec{\omega}=(\sin \vartheta \sin \psi \, \dot\varphi + \cos \psi \, \dot \vartheta) \vec e_1' + (\sin \vartheta \cos \psi \, \dot \varphi - \sin \psi \, \dot \vartheta) \vec e_2' + (\cos \vartheta \, \dot \varphi + \dot \psi) \vec e_3'$,
kjer je $\dot{\varphi}$ precesija vrtavke, $\dot{\vartheta}$ pa nutacija vrtavke. To vstavimo v Euler Lagrangevo enačbo, da ugotovimo, zveze med njimi. ($T=\sum \frac{1}{2} J_i \omega_i^2$) 
\subsection{Togo telo z 1 nepremično točko}

Po Eulerjevem izreku je gibanje takega telesa ob vsakem času rotacija okoli neke osi. \\
$\frac{d \rv}{dt} = \vec \omega \times \rv$ \medskip \\

$\vec L = \sum_i m_i \left[ (\rv_i \cdot \rv_i) \vec \omega - (\rv_i \cdot \vec \omega) \rv_i \right] = \underline L \vec \omega$ \\
$\underline L = \int \rho \, dV \begin{bmatrix}
y^2 + z^2 & -xy & -xz \\
-yx & x^2 + z^2 & -yz \\
-zx & -zy & x^2 + y^2 
\end{bmatrix}$ \\
$\vec L' = R \vec L$ in $\vec \omega ' = R \vec \omega$ $\implies$ $\underline{J}' = R \underline{L} R^T$ \\
$T = \frac{1}{2} \sum_i m_i v_i^2 = \frac{1}{2} \vec \omega \underline{J} \vec \omega$ \medskip \\

Delajmo v lastnem sistemu telesa. \\
$\frac{d \vec L}{dt} = \left( \frac{d \vec L}{dt} \right)_\mathrm{neinerc.} + \vec \omega \times \vec L = \vec M$ \\
$\left( \frac{d \vec L}{dt} \right)_\mathrm{neinerc.} = \sum_\alpha J_{\alpha \alpha} \dot \omega_\alpha \vec e_\alpha$ \medskip \\

\textbf{Eulerjeve enačbe} \\
$J_1 \dot \omega_1 - (J_2 - J_3) \omega_2 \omega_3 = M_1$ \\
$J_2 \dot \omega_2 - (J_3 - J_1) \omega_3 \omega_1 = M_2$ \\
$J_3 \dot \omega_3 - (J_1 - J_2) \omega_1 \omega_2 = M_3$ \medskip \\


\subsection{Vpeta osno simetrična vrtavka}

Uporabljamo Eulerjeve kote. \\
$J_1 = J_2 \neq J_3$ \medskip \\
Če zapišemo Lagrangevo funkcijo za tako vrtavko, dobimo, da je: \\
$L = \frac{1}{2}J_1(\omega_1^2 + \omega_2^2) + \frac{1}{2} J_3 \omega_3^2 - mgz = $\\
$\ \ \ = \frac{1}{2} J_1(\sin^2 \vartheta \, \dot \varphi^2 + \dot \vartheta^2) + \frac{1}{2}J_3(\dot \psi + \cos \vartheta \, \dot \varphi)^2 - mgl\cos \vartheta$ \medskip\\
Konstante gibanja so enake: \\
$p_\psi = J_3(\dot \psi + \cos \vartheta \, \dot \varphi) = J_3 \omega_3 = konst = J_1 a$ \\
$p_\varphi = (J_1 \sin^2 \vartheta + J_3 \cos^2 \vartheta) \dot \varphi + J_3 \cos \vartheta \, \dot \psi = konst. = J_1 b$ \medskip \\
Celotna energija pa se glasi: \\
$E = T + V = konst.$ \medskip \\

Odvisnosti kotov so: \\
$\dot \varphi = \frac{b - a\cos \vartheta}{\sin^2 \vartheta}$ \\
$\dot \psi = \frac{J_1}{J_3} a - \frac{b - a\cos \vartheta}{\sin^2 \vartheta} \cos \vartheta$ \medskip \\

\textbf{Osnovna enačba vrtavke} \\
$\tilde E = \frac{1}{2}J_1 \dot \vartheta^2 + \tilde V(\vartheta) = E - \frac{1}{2} J_3 \omega_3^2 =  konst.$\\
$\tilde V(\vartheta) = \frac{1}{2}\frac{(b - a \cos \vartheta)^2}{\sin^2 \vartheta} + mgl\cos \vartheta$ \\
$\implies$ $\dot \vartheta^2 = \frac{2}{J_1}(\tilde{E} - \tilde V(\vartheta))$ \medskip \\

Pogosto uporabna substitucija $u = \cos \vartheta$. \\
$\alpha = \frac{2 \tilde E}{J_1}$ \quad in \quad $\beta = \frac{2mgl}{J_1}$ \\
$t = \sqrt{\frac{J_1}{2}} \int_{\vartheta(0)}^{\vartheta(t)} \frac{d \vartheta}{\sqrt{\tilde E - \tilde V(\vartheta)}} = - \int_{u(0)}^{u(t)} \frac{du}{\sqrt{f(u)}}$ \\
$f(u) = (1 - u^2)(\alpha - \beta u) - (b - au)^2$ \\
S pogojem $f(u) \geq 0$ kvalitativno določimo obnašanje vrtavke. \medskip \\


\section{Hamiltonov formalizem}

Namesto s $q_i$ delamo s $p_i = \frac{\partial L}{\partial \dot q_i}$. \\
$\dot p_i = \frac{\partial L}{\partial q_i}$ \quad (iz EL enačbe) \medskip \\

\textbf{Hamiltonove enačbe} \\
$\dot q_i = \frac{\partial H}{\partial p_i}$ \\
$\dot p_i = - \frac{\partial H}{\partial q_i}$ \\
$\frac{\partial H}{\partial t} = - \frac{\partial L}{\partial t}$ \medskip \\


\section{Nabit delec v magnetnem polju}

V Lagrangevemu formalizmu lahko upoštevamo tudi potenciale $U$ za katere velja $F_i = - \frac{\partial U(q_i, \dot q_i, t)}{\partial q_i} + \frac{d}{dt} \frac{\partial U(q_i, \dot q_i, t)}{\partial \dot q_i}$. \medskip \\

$\vec B = \nabla \times \vec A$, \qquad $\vec E = -\nabla \varphi - \frac{\partial \vec A}{\partial t}$ \\
$U = e\varphi - e \vec v \cdot \vec A$ \\
$\vec F = e(\vec E + \vec v \times \vec B) = - \nabla_{\rv} U + \frac{d}{dt} \nabla_{\vv} U$ \medskip \\

$L = \frac{1}{2}mv^2 - U = \frac{1}{2}mv^2 - e\varphi + e \vec v \cdot \vec A$ \\
$\vec p = \nabla_{\vv} L = m \vec v + e \vec A$ \\
$H = \vec p \cdot \vec v - L = \frac{1}{2} m v^2 + e \varphi = \frac{|\vec p - e \vec A|^2}{2m} + e \varphi$ \medskip \\


\section{Poissonovi oklepaji}

$\{f, g\} = \sum_i \left( \frac{\partial f}{\partial q_i} \frac{\partial g}{\partial p_i} - \frac{\partial f}{\partial p_i} \frac{\partial g}{\partial q_i} \right)$ \\
$\{f, \lambda g + \mu h\} = \lambda \{f, g\} + \mu \{f, h\}$, \quad $\lambda, \mu \in \mathbb{C}$ \\
$\{f, g\} = - \{g, f\}$ \\
$\{f, gh\} = \{f, g\}h + \{f, h\} g$ \\
$\{f, \{g, h\}\} + \{g, \{h, f\}\} + \{h, \{f, g\}\} = 0$ \medskip \\

$\frac{df}{dt} = \{f, H\} + \frac{\partial f}{\partial t}$ \\
$f$ konstanta gibanja $\left( \frac{df}{dt} = 0 \right)$ $\implies$ $\{H, f\} = \frac{\partial f}{\partial t}$ \medskip \\

$\{q_i, q_j\} = 0$ \\
$\{p_i, p_j\} = 0$ \\
$\{q_i, p_j\} = \delta_{ij}$ \\
$\{l_i, l_j\} = \varepsilon_{ijk}l_k$ \medskip \\

\section{Majhna nihanja}

$L = \frac{1}{2} \sum_{ij} w_{ij}(\underline q) \dot q_i \dot q_j - V(\underline q)$ \\
Naj bo $\underline q^0$ ravnovesne lega sistema in $\underline \eta = \underline q - \underline q^0$. \\
$\implies$ $V(\underline q) = V(\underline q^0) + \frac{1}{2} \sum_{ij} \frac{\partial^2 V}{\partial q_i \partial q_j}\big|_{\underline{q}^0} \eta_i \eta_j + \cdots$ \\
$\implies$ $w_{ij}(\underline q) = w_{ij}(\underline q^0) + c\dots$ \medskip \\

Definiramo $T_{ij} = w_{ij}(\underline q_0)$ in $V_{ij} = \frac{1}{2} \sum_{ij} \frac{\partial^2 V}{\partial q_i \partial q_j}\big|_{\underline{q}^0}$. \\
$\implies$ $\tilde L = T - (V - V_0) = \frac{1}{2}(\underline{\dot \eta}^T \uunderline T \, \underline{\dot \eta} - \underline{\eta}^T \uunderline V \, \underline{\eta})$ \\
$\implies$ $\sum_j T_{ij} \ddot \eta_j + \sum_j V_j \eta_j = 0$ \medskip \\

Lastna nihanja $\eta_i = C \alpha_i e^{i \omega t}$ \\
$\implies$ $\uunderline V \, \underline a = \omega^2 \uunderline T \, \underline a$ \\
$\implies$ Lastne frekvence $\omega_k$ in lastni vektorji $\underline a_k$ ($\underline A$). \medskip \\

Splošno nihanje $\eta_i(t) = \sum_{k = 1}^{n} a_{ki} \alpha_k(t)$ \\
Izberemo $\underline a_k$ tako, da $\underline a_i^T \uunderline T \, \underline a_j = \delta_{ij}$ \\
$\implies$ $\uunderline A^T \uunderline T \, \uunderline A = \uunderline I$ in $\uunderline A^T \uunderline V \, \uunderline A = \mathrm{diag}(\lambda_1, \dots, \lambda_n)$ \\
$\implies$ $\alpha_k = C_k \cos(\omega_k t + \delta_k)$ \medskip \\


Reševanje nalog: \\
Zapišemo matriko za T in V in nato rešujemo enačbo
\begin{equation}
	\text{det}(\text{V}-\lamda T)=0
\end{equation}
Iz česar dobimo lastne vrednosti $\lambda_k$ in $\vec{a_k}$. Rešitev enačbe, pa je oblike
\begin{equation}
	\eta(t)=\sum \alpha_k(t)\vec{a_k}
\end{equation}
kjer je $\alpha_k$ rešitev posamezne enačbe $(A_k\sin(\lambda_k t)+B_k\cos(\lambda_k t))$


\section{Lagrangev formalizem za zvezno sredstvo}

$\mathcal{L}(u, u_x, u_t, t) = \mathcal{T} - \mathcal{V}$ \quad (lagrangian na volumen) \\
$\frac{d}{dt}\frac{\partial \mathcal{L}}{\partial u_t} + \frac{d}{dx} \frac{\partial \mathcal{L}}{\partial u_x} - \frac{\partial \mathcal{L}}{\partial u} = 0$ \\
$\pi = \frac{\partial \mathcal{L}}{\partial u_t}$ \quad (posplošeni impulz) \\
$H = \int (\pi u_t - \mathcal{L}) \, dV = \int \mathcal{H} \, dV$ \medskip \\

\section{Kanonične transformacije}

Transformacija $q_i \to Q_i(\underline q, \underline p, t)$ in $p_i \to P_i(\underline q, \underline p, t)$ je kanonična, če ohranja Poissonove oklepaje. Tedaj: \\
$\dot Q_i = \frac{\partial H}{\partial P_i} + \frac{\partial Q_i}{\partial t}$ \\
$\dot P_i = - \frac{\partial H}{\partial Q_i} + \frac{\partial P_i}{\partial t}$ \medskip \\

\section{Hamilton-Jacobijev formalizem}

$S = \int_{t_1}^{t_2}L(\underline q, \underline{\dot q}, t) \, dt$
$S(q_2, t_2)$ si predstavljamo kot funkcijo končne točke. Integral teče po klasični poti $q(t)$, ki ustreza gibalnim enačbam. \\
$\frac{\partial S}{\partial q} = p$ \quad (oziroma $\nabla S(\vec r) = \vec p$) \\
$H + \frac{\partial S}{\partial t} = 0$ \\
\section{Vektorski produkti}
Smer vektorja: $\dot{r}=\frac{\dot{\vec{r}}\vec{r}}{r}$ \\

\section{Diferencialne enačbe}
\textbf{Enačba dušenega nihanja:} \\
$$\dot{\dot{\phi}}+2\beta\dot{\phi}+\omega_0^2\phi=0$$
Nastavek = $\phi=A\cos(\omega t)e^{-\beta t}$ \quad ($\omega_0 $ in $\omega$ sta različni. \\ Nastavek odvajaš in ga vstaviš v enačbo, da ugotoviš $\omega$. A dobiš iz začetnih pogojev. \medskip \\
\textbf{Enačba z kompleksnim delom:}\\
$$\ddot{\zeta}+Ai \dot{\zeta}-B\zeta=0$$
Nastavek = $\zeta=\zeta_0 e^{-iut}$ \\
Nastavek odvajaš in vstaviš v enačbo, da ugotoviš u. $\zeta_0$ ugotoviš iz začetnih pogojev. \medskip \\
\textbf{Enačba hiperboličnega nihanja:}
$$\ddot{x}-\omega^2 x=0$$
Nastavek : $x=Ae^{\lambda t}$
\\ Nastavek odvajaš in ga vstaviš v enačbo, da ugotoviš $\lambda$. A dobiš iz začetnih pogojev. \medskip \\

\section{Enačbe ploskev}
\textbf{Elipsa:}\\
$x^2(1+b)+y^2(1-b)=a \quad \frac{x^2}{a^2}+\frac{y^2}{b^2}=1$
\textbf{Hiperbola:}\\
$x^2(1+b)-y^2(1-b)=a \quad \frac{x^2}{a^2}-\frac{y^2}{b^2}=1$
\textbf{Parabola:}\\
$y^2=2px$
\end{multicols}
\end{document}