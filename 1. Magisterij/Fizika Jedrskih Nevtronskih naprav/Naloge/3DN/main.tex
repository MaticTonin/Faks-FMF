\documentclass[slovene,11pt,a4paper]{article}
\usepackage[pdftex]{graphicx}
\DeclareGraphicsExtensions{.pdf,.png}

\usepackage{siunitx}
\usepackage{tikz}
\usepackage{float}
\usepackage[margin=1cm,bottom=2cm,foot=1cm]{geometry}
\usepackage{fullpage}
\usepackage{a4wide}
\setlength{\parindent}{0pt}
\setlength{\parskip}{0.5ex}
\usepackage{amsmath}
\usepackage{amsfonts}
\usepackage{mathrsfs}
\usepackage[usenames]{color}
\usepackage[utf8]{inputenc}
\usepackage{siunitx}
\usepackage{caption}

\DeclareCaptionType{equ}[][]
\def\phi{\varphi}
\def\eps{\varepsilon}
\def\theta{\vartheta}

\newcommand{\thisyear}{2020/21}

\renewcommand{\Re}{\mathop{\rm Re}\nolimits}
\renewcommand{\Im}{\mathop{\rm Im}\nolimits}
\newcommand{\Tr}{\mathop{\rm Tr}\nolimits}
\newcommand{\diag}{\mathop{\rm diag}\nolimits}
\newcommand{\dd}{\,\mathrm{d}}
\newcommand{\ddd}{\mathrm{d}}
\newcommand{\ii}{\mathrm{i}}
\newcommand{\lag}{\mathcal{L}\!}
\newcommand{\ham}{\mathcal{H}\!}
\newcommand{\four}[1]{\mathcal{F}\!\left(#1\right)}
\newcommand{\bigO}[1]{\mathcal{O}\!\left(#1\right)}
\newcommand{\sh}{\mathop{\rm sinh}\nolimits}
\newcommand{\ch}{\mathop{\rm cosh}\nolimits}
\renewcommand{\th}{\mathop{\rm tanh}\nolimits}
\newcommand{\erf}{\mathop{\rm erf}\nolimits}
\newcommand{\erfc}{\mathop{\rm erfc}\nolimits}
\newcommand{\sinc}{\mathop{\rm sinc}\nolimits}
\newcommand{\rect}{\mathop{\rm rect}\nolimits}
\newcommand{\ee}[1]{\cdot 10^{#1}}
\newcommand{\inv}[1]{\left(#1\right)^{-1}}
\newcommand{\invf}[1]{\frac{1}{#1}}
\newcommand{\sqr}[1]{\left(#1\right)^2}
\newcommand{\half}{\frac{1}{2}}
\newcommand{\thalf}{\tfrac{1}{2}}
\newcommand{\pd}{\partial}
\newcommand{\Dd}[3][{}]{\frac{\ddd^{#1} #2}{\ddd #3^{#1}}}
\newcommand{\Pd}[3][{}]{\frac{\pd^{#1} #2}{\pd #3^{#1}}}
\newcommand{\avg}[1]{\left\langle#1\right\rangle}
\newcommand{\norm}[1]{\left\Vert #1 \right\Vert}
\newcommand{\braket}[2]{\left\langle #1 \vert#2 \right\rangle}
\newcommand{\obraket}[3]{\left\langle #1 \vert #2 \vert #3 \right \rangle}
\newcommand{\hex}[1]{\texttt{0x#1}}

\renewcommand{\iint}{\mathop{\int\mkern-13mu\int}}
\renewcommand{\iiint}{\mathop{\int\mkern-13mu\int\mkern-13mu\int}}
\newcommand{\oiint}{\mathop{{\int\mkern-15mu\int}\mkern-21mu\raisebox{0.3ex}{$\bigcirc$}}}

\newcommand{\wunderbrace}[2]{\vphantom{#1}\smash{\underbrace{#1}_{#2}}}

\renewcommand{\vec}[1]{\overset{\smash{\hbox{\raise -0.42ex\hbox{$\scriptscriptstyle\rightharpoonup$}}}}{#1}}
\newcommand{\bec}[1]{\mathbf{#1}}




\newcommand{\Ai}{\mathrm{Ai}}
\newcommand{\Bi}{\mathrm{Bi}}
\newcommand{\bi}[1]{\hbox{\boldmath{$#1$}}}
\newcommand{\bm}[1]{\hbox{\underline{$#1$}}}

\title{1. Izračun Airyjevih funkcij}
\author{Matic Tonin - 28181098 }
\date{Oktober 2020}

\begin{document}

\begin{center}
\thispagestyle{empty}
\parskip=14pt%
\vspace*{3\parskip}%
\begin{Huge}Širjenje nevtronov v prostoru\end{Huge}


3. domača naloga v sklopu predmeta \\
Fizike jedrskih nevtronskih naprav

Avtor:

Matic Tonin

Vpisna številka: 28181098

Profesor: prof. dr. Luka Snoj

Asistent: doc. dr. Andrej Zohar


\rule{7cm}{0.4pt}

Pod okvirom:

FAKULTETE ZA FIZIKO IN MATEMATIKO, LJUBLJANA

Akademsko leto 2021/2022


\end{center}
\pagebreak
\section*{Naloga}
V velikem bazenu z vodo je krogla iz polietilena s polmerom 50 cm. V središču krogle je točkast izotropen izvor nevtronov. Kakšno je razmerje med fluksom na površini in na polovici radija krogle? Uporabi enogrupni difuzijski približek in podatke
\begin{equation}
\begin{array}{lll} 
& \Sigma_{a}\left(\mathrm{~cm}^{-1}\right) & D(\mathrm{~cm}) \\
\text { polietilen } & 0.010 & 4.0 \\
\text { voda } & 0.020 & 2.0
\end{array}
\end{equation}
\section{Reševanje}
Če uporabimpo našo difuzijsko enačbo, bo veljalo, da je 
\begin{equation}
    \nabla \psi=\frac{\sum_a}{D}\psi
\end{equation}
kar v sferičnih kooridnatah rešimo kot 
\begin{equation}
     \frac{1}{r^2}\frac{\partial}{\partial r}\left(r^2\psi\right)=-\frac{\sum_a}{D}\upsilon
\end{equation}
Ker pa je reševanje take enačbe zelo zahtevno, si raje vzamemo pogoj, da je 
\begin{equation}
    \psi=\psi_1 \quad \text{if } 0<r<R \qquad \psi=\psi_2 \quad \text{if } r>R
\end{equation}
kjer sta
\begin{equation}
    \psi_1=\frac{Ae^{-\frac{r}{L_1}}+Be^{\frac{r}{L_1}}}{r} \qquad \psi_1=\frac{Ce^{-\frac{r}{L_2}}+De^{\frac{r}{L_2}}}{r}
\end{equation}
Sedaj pa si poglejmo robne in začetne pogoje našega nastavka
\begin{enumerate}
    \item $\lim_{r \rightarrow \infty} \psi < \infty$ iz česar sledi, da je $D=0$ \\
    \item $\lim_{r \rightarrow 0} \psi =S$ iz česar sledi, da je $S=D_1 4\pi(A+B)$ \\
    \item Zveznost fluska na meji $\psi_1(R)=\psi_2(R)$ \\
    \item $j_1(R)=j_2(R)$ iz česar je $D_1\frac{\partial \psi_1}{\partial r}=D_2\frac{\partial \psi_2}{\partial r}$
\end{enumerate}
Če uporabimo pogoj 3. in 4. dobimo, da je 
\begin{equation}
    \text{3.}\qquad Ae^{-\frac{R}{L_1}}+Be^{\frac{R}{L_1}}=Ce^{-\frac{R}{L_2}} 
\end{equation}
\begin{equation}
    \text{4.}\qquad -A\frac{e^{-\frac{R}{L_1}}}{R^2}-A\frac{e^{-\frac{R}{L_1}}}{RL_1}-B\frac{e^{\frac{R}{L_2}}}{R^2}+B\frac{e^{\frac{R}{L_1}}}{RL_1}=-C\frac{e^{-\frac{R}{L_2}}}{R^2}-C\frac{e^{-\frac{R}{L_2}}}{RL_2}
\end{equation}
Z reševanjem sistema teh dve enačb pa bi dobili rešitev ki je oblike
\begin{equation}
    A=\frac{S}{4\pi D_1} \qquad B=-\frac{S}{4\pi D_1}\left(\frac{\frac{1}{RL_1\sinh(\frac{R}{L_1}}+\left(\frac{D_2}{D_1}\left(\frac{1}{R^2}+\frac{1}{RL_2}-\frac{1}{R^2}\right)\cosh\left(\frac{R}{L_1}\right)\right)}{\frac{1}{RL_1\cosh(\frac{R}{L_1}}+\left(\frac{D_2}{D_1}\left(\frac{1}{R^2}+\frac{1}{RL_2}-\frac{1}{R^2}\right)\sinh\left(\frac{R}{L_1}\right)\right)}\right)
\end{equation}
in 
\begin{equation}
    C=\frac{e^{-\frac{R}{L_2}}}{A\cosh\left(\frac{R}{L_1}\right)+B\sinh\left(\frac{R}{L_1}\right)}
\end{equation}
V resnici pa nas bosta zanimala le koeficienta A in B.
V primeru, ko računamo fluks na polovici krogle in na površini, bi lahko definirali, da je iskana količina kar
\begin{equation}
    \nu=\frac{\psi_1(R)}{\psi_1(R/2)}
\end{equation}
Z vstavljanjem znanja o razdaljah in koeficientih $D_1$ in $D_2$ pa bi dobili, da je 
\begin{equation}
    \nu=\frac{\psi_1(R)}{\psi_1(R/2)}=0.158
\end{equation}


\end{document}

